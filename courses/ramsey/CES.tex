
\documentclass[11pt,a4paper,english]{article}

\usepackage[utf8]{inputenc}
\usepackage[margin=1in]{geometry}
\usepackage{changepage}
\usepackage{pdflscape}
\usepackage{natbib}
\setlength{\bibsep}{0.0pt}
\usepackage{hyperref}
\usepackage{amsmath, amsthm, amssymb}
\usepackage{multirow}
\usepackage[singlespacing]{setspace}
\usepackage[english]{babel}
\usepackage{microtype}
\usepackage{graphicx}
\usepackage{booktabs}
\usepackage{longtable}
\usepackage{soul,color}
\usepackage{authblk}
\usepackage{array}
\usepackage{subcaption}
\usepackage{afterpage}
\usepackage[bottom]{footmisc}

\newtheorem{prop}{Proposition}
\newtheorem{cor}{Corollary}

\setlength{\abovedisplayskip}{2.5pt}
\setlength{\belowdisplayskip}{2.5pt}

% Commands from http://www.jwe.cc/2012/03/stata-latex-tables-estout/
% *****************************************************************
% Estout related things
% *****************************************************************
\newcommand{\sym}[1]{\rlap{#1}} 
\let\estinput=\input% define a new input command so that we can still flatten the document
\newcommand{\estwide}[3]{
		\vspace{.75ex}{
			\begin{tabular*}
			{\linewidth}{@{\hskip\tabcolsep\extracolsep\fill}l*{#2}{#3}}
			\toprule
			\estinput{#1}
			\bottomrule
			%\addlinespace[.75ex]
			\end{tabular*}
			}
		}	

\newcommand{\estauto}[3]{
	\vspace{.75ex}{
		\begin{tabular}{l*{#2}{#3}}
		\toprule
		\estinput{#1}
		\bottomrule
		%\addlinespace[.75ex]
		\end{tabular}
		}
	}

		
% Allow line breaks with \\ in specialcells
	\newcommand{\specialcell}[2][c]{%
	\begin{tabular}[#1]{@{}c@{}}#2\end{tabular}}

% *****************************************************************
% Custom subcaptions
% *****************************************************************
% Note/Source/Text after Tables
\newcommand{\figtext}[1]{
	\captionsetup{justification=justified,font=footnotesize}
	\caption*{\hspace{6pt}\hangindent=1.5em #1}
	}
\newcommand{\fignote}[1]{\figtext{\emph{Note:~}~#1}}

\newcommand{\figsource}[1]{\figtext{\emph{Source:~}~#1}}

\newcommand{\figsummary}[1]{\figtext{\emph{Summary:~}~#1}}

% Add significance note with \starnote
\newcommand{\starnote}{
	\vspace{-1.9ex}
	\captionsetup{justification=justified, singlelinecheck=false, font=footnotesize}
	\caption*{\hspace{6pt}\hangindent=1.5em 
	\figtext{${}^{*}\, p < 0.1$, ${}^{**}\, p < 0.05$, ${}^{***}\, p < 0.01$. Standard errors in parentheses.}}
}

% Stars
\def\stars{${}^{*}\, p < 0.1$, ${}^{**}\, p < 0.05$, ${}^{***}\, p < 0.01$.}

% *****************************************************************
% siunitx
% *****************************************************************
\usepackage{siunitx} % centering in tables
%	\sisetup{
%		detect-mode,
%		tight-spacing		= true,
%		group-digits		= false ,
%		input-signs		= ,
%		input-symbols		= ( ) [ ] - + *,
%		input-open-uncertainty	= ,
%		input-close-uncertainty	= ,
%		table-align-text-post	= false
%        }
\sisetup{parse-numbers = false}

\title{Some additional notes}
\author[1]{\`{E}ric Roca Fern\'{a}ndez}
\date{}



\begin{document}

\def\sym#1{\ifmmode^{#1}\else\(^{#1}\)\fi}

\maketitle

\section{Interest rates and wages using the CES production function}

Suppose that $F(K,L) = (\alpha K^\rho + (1-\alpha) L^\rho)^\frac{1}{\rho}.$
We can compute the interest rate and wages in the intensive form using several alternatives:

\subsection{Derive the function $F$}

First compute the derivative with respect to capital:

\begin{align*}
    r = F_{K}(K,L) =& \frac{1}{\rho}\left(\alpha K^\rho + (1-\alpha) L^\rho\right)^\frac{1-\rho}{\rho}\alpha \rho K^{\rho-1} \\
    =& \alpha \left( \alpha K^\rho + (1-\alpha)L^\rho \right)^\frac{1-\rho}{\rho} K^{\rho-1} = \\
    =& \alpha \left( \alpha K^\rho + (1-\alpha)L^\rho \right)^\frac{1-\rho}{\rho} \underbrace{\frac{K^{\rho-1}}{L^{\rho-1}}}_{=k^{\rho-1}}L^{\rho-1} = \\
    =& \alpha \left( \underbrace{L^{\frac{\rho-1}{1-\rho}\rho}}_{=L^{-\rho}}(\alpha K^\rho + (1-\alpha)L^\rho) \right)^\frac{1-\rho}{\rho}k^{\rho-1} = \\
    =& \alpha \left( \alpha \frac{K^\rho}{L^\rho} + (1-\alpha)\frac{L^\rho}{L^\rho} \right)^\frac{1-\rho}{\rho}k^{\rho-1} = \\
    =& \alpha \left(\alpha k^\rho + (1-\alpha) \right)^\frac{1-\rho}{\rho}k^{\rho-1}.
\end{align*}

Since $f(k) = \left(\alpha k^\rho + (1-\alpha) \right)^\frac{1}{\rho}$, then:

\begin{align*}
    r &= \alpha \left(\alpha k^\rho + (1-\alpha) \right)^\frac{1-\rho}{\rho}k^{\rho-1} = \\
     &= \alpha \left(\alpha k^\rho + (1-\alpha) \right)^\frac{1}{\rho} \left(\alpha k^\rho + (1-\alpha)\right)^{-1} k^{\rho-1} = \\
     &= \alpha f(k) \left( \alpha k^\rho + (1-\alpha) \right)^\frac{-\rho}{\rho} k^{\rho-1} = \\
     &= \alpha f(k) f(k)^{-\rho} k^{\rho-1} = \\
     &= \alpha f(k)^{1-\rho} k^{\rho-1} = \\
     &= \alpha \left( \frac{f(k)}{k}\right)^{1-\rho}.
\end{align*}

Next, we do the same for labour:

\begin{align*}
    w = F_{L}(K,L) =& \frac{1}{\rho}\left( \alpha K^\rho + (1-\alpha) L^\rho \right)^\frac{1-\rho}{\rho} (1-\alpha)\rho L^{\rho-1} = \\
    =& (1-\alpha) \left(\alpha K^\rho + (1-\alpha)L^\rho \right)^\frac{1-\rho}{\rho} L^{\rho-1} = \\
    =& (1-\alpha) \left( \underbrace{L^{(\rho-1)\frac{\rho}{1-\rho}}}_{=L^{-\rho}} \left( \alpha K^\rho + (1-\alpha) L^\rho \right)\right)^\frac{1-\rho}{\rho} = \\
    =& (1-\alpha) \left( \alpha \frac{K^\rho}{L^\rho} + (1-\alpha) \frac{L^\rho}{L^\rho}\right)^\frac{1-\rho}{\rho} = \\
    =& (1-\alpha) \left(\alpha k^\rho + (1-\alpha) \right)^\frac{1-\rho}{\rho}.
\end{align*}

We proceed as before:

\begin{align*}
    w =& (1-\alpha)\left(\alpha k^\rho + (1-\alpha)\right)^\frac{1-\rho}{\rho} = \\
    =& (1-\alpha)\left(\alpha k^\rho + (1-\alpha)\right)^\frac{1}{\rho}\left(\alpha k^\rho + (1-\alpha)\right)^\frac{-\rho}{\rho} = \\
        =& (1-\alpha)f(k)f(k)^{-\rho} = \\
        =& (1-\alpha)f(k)^{1-\rho}.
    \end{align*}

\subsection{Use the intensive forms}

In that case, first write the production function in intensive terms and compute the derivatives.
Since $F$ is homogenous of degree one, we have that:

$$F\left( \frac{K}{L}, \frac{L}{L} \right) = \frac{1}{L} F(K,L).$$
Also, we define

$$f(k) = F\left(\frac{K}{L}, 1\right).$$

Hence, $$F(K,L) = L f(k) = L f\left(\frac{K}{L}\right).$$

We have a CES production function that satifies:

\begin{align*}
    F(K,L) &= (\alpha K^\rho + (1-\alpha)L^\rho)^\frac{1}{\rho} \\
    \frac{1}{L} F(K,L) &= F\left(\frac{K}{L},1\right) =\\
    &= F\left( \alpha \left(\frac{K}{L}\right)^\rho + (1-\alpha) \left(\frac{L}{L}\right)^\rho\right)^\frac{1}{\rho} = \\
    &= \left(\alpha k^\rho + (1-\alpha)\right)^\frac{1}{\rho}.\\
    &\mathrm{or} \\
    \frac{1}{L} F(K,L) &= \frac{1}{L} \left(\alpha K^\rho + (1-\alpha)L^\rho\right)^\frac{1}{\rho} = \\
    &= \left(L^{-\rho}\left(\alpha K^\rho + (1-\alpha)L^\rho\right)\right)^\frac{1}{\rho} = \\
    &= \left(\alpha k^\rho + (1-\alpha)\right)^\frac{1}{\rho}.
\end{align*}

From here, we can compute the derivatives as we did before:

\begin{align*}
    r = F_K (K,L) =& \frac{\partial  L f\left( \frac{K}{L} \right)}{\partial K} = L f^\prime\left(\frac{K}{L}\right) \frac{1}{L} = \\
    =& f^\prime\left(\frac{K}{L}\right) = \\
    =& f^\prime(k).
\end{align*}

Using $ f(k) = \left( \alpha k^\rho + (1-\alpha)\right)^\frac{1}{\rho}$ we get:

\begin{align*}
    r =& f^\prime(k) = \\
=& \frac{1}{\rho}\left(\alpha k^\rho + (1-\alpha)\right)^\frac{1-\rho}{\rho}\alpha \rho  k^{\rho-1} = \\
=& \alpha \left(\frac{f(k)}{k}\right)^{1-\rho}.
\end{align*}

For wages we have:

\begin{align*}
    w = F_L(K,L) =& \frac{\partial L f\left( \frac{K}{L} \right)}{\partial L} = f\left(\frac{K}{L}\right) + f^\prime\left(\frac{K}{L}\right)\frac{-K}{L^2}L  = \\
    =& f(k) - f^\prime(k) k.
\end{align*}

Substituting $f(k) = \left(\alpha k^\rho + (1-\alpha)\right)^\frac{1}{\rho}$:

\begin{align*}
    w =& f(k) - f^\prime(k) k = \\
    =&\left( \alpha k^\rho + (1-\alpha)\right)^\frac{1}{\rho}  - \frac{1}{\rho}\left( \alpha k^\rho + (1-\alpha)\right)^\frac{1-\rho}{\rho}\alpha \rho k^{1-\rho} k^\rho = \\
    =&\left( \alpha k^\rho + (1-\alpha)\right)^\frac{1}{\rho}  - \alpha\left( \alpha k^\rho + (1-\alpha)\right)^\frac{1-\rho}{\rho}k^\rho = \\
    =& f(k) - \alpha\left( \alpha k^\rho + (1-\alpha)\right)^\frac{1}{\rho}\left( \alpha k^\rho + (1-\alpha)\right)^{-1} k^\rho = \\
    =& f(k) - \alpha f(k)f(k)^{-\rho}k^\rho = \\
    =& f(k) - \alpha f(k)^{1-\rho}k^\rho = \\
    =& f(k)^{1-\rho}\left(f(k)^\rho - \alpha k^\rho\right) = \\
    =& f(k)^{1-\rho}\left(\alpha k^\rho + (1-\alpha) - \alpha k^\rho\right) = \\
    =& f(k)^{1-\rho}(1-\alpha).
\end{align*}


\section{The Jacobian matrix using total differentiation}

The two dynamic equations of the Ramsey model are:

\begin{align*}
    u^\prime(c_{t}) =& \beta(f^\prime(k_{t+1}) + 1 - \delta) u^\prime(c_{t+1}),\\
    k_{t+1} =& f(k_{t}) + (1-\delta)k_{t} - c_{t}.
\end{align*}

We are interested in computing $\frac{\mathrm{d} c_{t+1}}{\mathrm{d} c_t}.$

Take the first equation of the dynamic system and differentiate it with respect to $c_t$ and $c_{t+1}$, noting that $k_{t+1}$ depends on $c_{t}.$

\begin{equation*}
    u^{\prime \prime}(c_t)\mathrm{d}c_t = \beta \overbrace{f^{\prime \prime}(k_{t+1})(-1)}^{k_{t+1}\, \mathrm{has}\, c_t}u^\prime(c_{t+1})\mathrm{d}c_t + \beta(f^\prime(k_{t+1}) + 1 - \delta)u^{\prime \prime}(c_{t+1})\mathrm{d}c_{t+1}.
\end{equation*}

Hence:

\begin{equation*}
    \frac{\mathrm{d}c_{t+1}}{\mathrm{d}c_t} = \frac{u^{\prime \prime}(c_t) + \beta f^{\prime \prime}(k_{t+1}) u^\prime (c_{t+1})}{\beta \left(f^\prime (k_{t+1}) + 1 - \delta \right) u^{\prime \prime}(c_{t+1})}.
\end{equation*}

Similarly, we can differentiate with respect to $c_{t+1}$ and $k_t$:

\begin{equation*}
    0 = \beta u^\prime (c_{t+1})f^{\prime \prime}(k_{t+1})\left(f^\prime (k_t)+1-\delta\right)\mathrm{d}k_t + \beta\left(f^\prime(k_{t+1}) + 1 - \delta \right)u^{\prime \prime}(c_{t+1})\mathrm{d}c_{t+1}.
\end{equation*}

Hence:

\begin{equation*}
    \frac{\mathrm{d}c_{t+1}}{\mathrm{d}k_t} = -\frac{\beta u^\prime(c_{t+1})f^{\prime \prime}(k_{t+1})\left(f^\prime(k_t)+1-\delta\right)}{\beta \left( f^\prime(k_{t+1})+1-\delta\right)u^{\prime \prime}(c_{t+1})}.
\end{equation*}

\section{Notes on the phase diagram}

The Ramsey model represents the economy using a system of first-order difference equations.

\begin{align*}
    u^\prime(c_t) =&\beta (f^\prime(k_{t+1}) + 1 - \delta) u^\prime(c_{t+1}) \\
    k_{t+1} =& f(k_t) + (1-\delta)k_t - c_t.
\end{align*}

In the phase diagram, we represent the combinations of $k$ and $c$ such that the either capital or consumption remain constant.

\subsection{The dyamics of $c$}

The first equation describes the dynamics of consumption in the economy.
If consumption has to remain constant, this implies that $c_t = c_{t+1}.$
Using this information we can compute the level of capital such that consumption is constant.
In fact,

\begin{align*}
    u^\prime(c_t) =& \beta (f^\prime(k_{t+1}) + 1 - \delta)\overbrace{u^\prime(c_t)}^{c_t=c_{t+1}} \implies\\
    1 =& \beta(f^\prime(k_{t+1}) + 1 - \delta).
\end{align*}

Denote $\bar{k}$ the level of capital that satifies this equality.
Suppose that $k > \bar{k}.$
Then, $f^\prime(k) < f^\prime(\bar{k}).$
Therefore, going back to the dynamic equation we have

\begin{align*}
    \frac{u^\prime(c_t)}{u^\prime(c_{t+1})} <& 1 \implies u^\prime(c_t) < u^\prime(c_{t+1}) \implies \\
        \implies c_t > c_{t+1}& \, \mathrm{or}\, c_{t+1} < c_t.
\end{align*}

Consequently, consumption falls over time.
Digramatically, we represent it indicating that for $k > \bar{k}$, consumption falls.
In the diagram, notice that the condition $1 = \beta(f^\prime(k_{t+1}) + 1 - \delta)$ is a vertical line:

\begin{align*}
    1 =& \beta (f^\prime(\bar{k}) + 1 - \delta) \\
    f^\prime(\bar{k}) =& \underbrace{\frac{1}{\beta}}_{>1}+\delta-1.
\end{align*}

Since $f^\prime$ is a decreasing function with $\lim_{k\rightarrow 0}f^\prime(k) = +\infty$ and $\lim_{k\rightarrow \infty}f^\prime(k) = 0,$ there is only one solution to this equation.
In the phase, this information is represented as follows:

\vspace{1em}
\begin{minipage}{0.45\textwidth}
\centering
\includegraphics[width=\linewidth]{"/home/eric/eric-roca.github.io/static/img/ramsey/dynamics_c"}
\end{minipage}
\hfill
\begin{minipage}{0.45\textwidth}
\centering
\includegraphics[width=\linewidth]{"/home/eric/eric-roca.github.io/static/img/ramsey/dynamics_g"}
\end{minipage}

\subsection{The dynamics of $k$}

We can proceed similarly for capital, noting that the relevant equation that determines its dyamic behaviour is:

\begin{equation*}
    k_{t+1} = f(k_t) + (1-\delta)k_t - c_t.
\end{equation*}

As before, if capital has to remain constant it implies that $k_t = k_{t+1}$, and impossing this on the equation above allows us to compute the combinations of capital and consumption such that the former is constant.
Hence, the level of consumption that implies a constant level of capital ($\bar{c}$) is given by:

\begin{equation*}
    \bar{c} = f(k) - \delta k.
\end{equation*}

Notice that this function is inversely U-shaped, with one maximum:

\begin{align*}
    \bar{c}^\prime =& f^\prime(k) - \delta, \\
    \bar{c}^{\prime \prime} =& f^{\prime \prime}(k) < 0
\end{align*}

From the first line, we have a maximum at:

\begin{equation*}
    f^\prime(k) - \delta = 0 \implies f^\prime(k) = \delta.
\end{equation*}

Notice that this maximum corresponds to a level of capital \emph{higher} that the one that ensures $c_t = c_{t+1}.$
In fact, $c_t = c_{t+1}$ implies $f^\prime(k) = \frac{1}{\beta}+\delta-1 > \delta$ meaning that such level is smaller than the one maximising $\bar{c}.$

Finally, we analyse the evoluton of capital.
Write the change in the level of capital as:

\begin{equation*}
    k_{t+1} - k_t = f(k_t) + (1-\delta)k_t - k_t - c_t = f(k_t) - \delta k_t -c_t.
\end{equation*}

For a consumption level above the one that keeps capital constant we  have:
\begin{equation*}
    k_{t+1} - k_t < 0 \implies k_{t+1} < k_t
\end{equation*}

and capital decreases.

\subsection{The phase diagram}

We can finally combine together the two previous graphs and obtain the phase diagram.
Using it, we know the dynamics of capital and consumption at any given point.
The intersection of the two curves represents the steady state: it is the combination of capital and consumption such that both variables remain constant.

Finally, we know that this model features a saddle-path.
This means that all inital combinations of capital and consumption diverge, except for one unique combination that converges towards the steady state.
The following figure illustrates this idea:

\vspace{1em}
\begin{minipage}{0.45\textwidth}
\centering
\includegraphics[width=\linewidth]{"/home/eric/eric-roca.github.io/static/img/ramsey/phase_diagram"}
\end{minipage}
\hfill
\begin{minipage}{0.45\textwidth}
\centering
\includegraphics[width=\linewidth]{"/home/eric/eric-roca.github.io/static/img/ramsey/dynamics_global"}
\end{minipage}







\end{document}
